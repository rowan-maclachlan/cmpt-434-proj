\documentclass[12pt]{report}
\usepackage[T1]{fontenc}  % Font encoding
\usepackage{mathptmx}     % Choose Times font 
\usepackage{microtype}    % Improves line breaks      
\usepackage{blindtext}    % Filler text 
\usepackage{hyperref}     % for urls with /url{}

%Margin --- 1 inch on all sides
\usepackage[letterpaper]{geometry}
\usepackage{times}
\usepackage{xcolor}
\usepackage{listings}
\usepackage{graphicx}
\usepackage{xparse}
\usepackage[font=small,labelfont=bf]{caption}
\NewDocumentCommand{\codeword}{v}{%
	\texttt{\textcolor{blue}{#1}}%
}
\geometry{top=1.0in, bottom=1.0in, left=1.0in, right=1.0in}

\usepackage{indentfirst} % Indent the first paragraph after section headings
\setlength{\parskip}{1em} % Set inter-paragraph spacing
\setlength{\parindent}{2em} % Set paragraph indentation
\usepackage{titlesec}
\usepackage{etoolbox}
\makeatletter
\patchcmd{\thebibliography}{%
    \chapter*{\bibname}\@mkboth{\MakeUppercase\bibname}{\MakeUppercase\bibname}}{%
    \section{References}}{}{}
\makeatother
%%%%%%%%%%%%%%%%%%%%%%%%%%%%%%%%%%%%%%%%%%%%%%%%%%%%%%%%%%%%%%%%%%%
% Title page
%%%%%%%%%%%%%%%%%%%%%%%%%%%%%%%%%%%%%%%%%%%%%%%%%%%%%%%%%%%%%%%%%%%

\begin{document}
	
	\titlepage{
		\vspace*{150px} % Push down title
		\centering % Centre title
		{\large \sc A description of project concepts, milestones, and goals\\}
		\vspace*{30px} % Push down sub text
		\hrulefill \\
		\vspace*{20px}
		{\Huge \uppercase{Project Proposal\\}} % Title
		\vspace*{20px}
		\hrulefill \\
		\vspace*{30px}
		February 27, 2019\\
		\vspace*{180px}
		
		\raggedright
		\hfill
		\textit{CMPT 434: Computer Networks} \\
		\vspace*{8px}
	    Christopher	Mykota-Reid
		\hfill
		Derek Eager\\
		Rowan MacLachlan\\
	}
	
	\pagebreak % Put title on its own page
	
	\tableofcontents
	
	\pagebreak
	
    \section{Proposal Requirements:}
    For an implementation project, you should:
    \begin{itemize}
        \item Describe what concept or issue you intend to study. 
        \item Describe the objectives of your investigation. 
        \item Give a brief plan of what you will do to investigate the issue. 
        \item Describe expected measurements or criteria by which you intend to
            evaluate whether you've succeeded at your objective.
    \end{itemize}

    \section{Core Concepts}
    What does peer-to-peer file sharing\cite{bittorrent-wiki}, blockchain,
    crypto-currencies\cite{ethereum-rlp}, private peer-to-peer voice and text
    messaging apps\cite{tox-site}, distributed social-media
    platforms\cite{distributed-social-networks}, and interplanetary internet
    models\cite{ipfs-site} have in common?  If you guessed: "Well, they all rely
    upon or implement in some way a distributed data structure to achieve some
    of their core functionality," you would be correct.  Most of the
    technologies from the above list implement or utilize a distributed hash
    table (or a similar distributed data structure, in the case of blockchain).
    These DHTs are similar in principle to a regular hash table in that they
    provide the structure for fast retrieval of an object by referencing the
    location of the object's hash value.  However, a distributed hash table is
    stored across nodes in a network, and this introduces a variety of
    complications involving node-lookup (finding the location of the stored data
    from its hash in the network), ensuring data redundancy (what happens if a
    node goes offline?), and data retrieval.  Although this technology is not
    \textit{new} by any standard, its continued use and development in various
    fields of application makes it a suitable candidate for an implementation
    project.
    
    \section{Objectives}
    We hope that by implementing a very simple version of a DHT and by demonstrating
    its use on a small network we will become familiar with the issues
    overcome by the more large-scale applications of DHTs.  As referenced above,
    these issues are myriad and often difficult to solve, but not impossible.
    By implementing an incremental plan to gradually roll out more complex features, new
    features, or re-implement sub-systems of our application, we will minimize
    risk so as to end up with a viable end-product for our presentation, while
    also taking full advantage of every learning opportunity which presents
    itself.

    \section{Directives}
    The rough plan for our implementation project will be as follows:
    \begin{enumerate}
        \item Collect and collate relevant technical (algorithm descriptions,
            source-code) and theoretical sources (papers) for the implementation
            of DHTs.  This has already begun.
        \item Divide the DHT into aspects.  Identify concerns such as hashing,
            security, node discovery, redundancy, etc.  This has already begun.
        \item Once the separate issues and their complexities are understood,
            design a minimum-viable product for our implementation.  This MVP
            should be modular and take advantage of as many pre-written
            libraries as are available, while ignoring any details of a DHT
            beyond its barest and least robust implementation.  It should also
            take into consideration the test application we have in
            mind.\footnote{Although it may not result in the most general implementation, it is
            more important that the core concepts are applied
            \textit{successfully} in our particular application.}
        \item Once the MVP has been designed, identify aspects of it which can
            be improved.  This will include things such as data redundancy
            (storing hashed data at multiple nodes), caching, and node
            deletion and addition (how is content from deleted nodes rehashed,
            and how is load redistributed on node addition?).
        \item Implement the MVP, keeping in mind the fact that a
            modular design will make feature improvement and addition easier
            than it would be in a poorly constructed monolithic design.
        \item Once the MVP is implemented and we move onto more advanced
            features, we have to consider the metrics and factors by which we
            measure the effectiveness of our implementation.  What are its
            drawbacks and restrictions?  How efficient is it at finding data on
            the overweb (participating nodes on the internet)?  How efficiently is
            data stored and updated?
    \end{enumerate}
    Naturally, the details of the implementation steps will evolve as our
    understanding of the challenges deepens, so the initial steps of the plan
    are the most important.

    \section{Measurements and Criteria}
    As referenced in the implementation plan, there are quantitative
    measurements available when considering the success or failure of our
    implementation.  Comparing these values to those achieved by more mature DHT
    implementations will provide valuable insight into the challenges posed by
    DHTs, and should be included in our final report.  However, the most
    valuable metric will be the realization of a working MVP.  This may be
    something as simple as the observed and correct distributed storage of one or more files on the
    personal computers of the researchers (Chris and Rowan) or a more glorious
    success - or, perhaps, the real success is the friends we made along the
    way.

%    \url{https://en.wikipedia.org/wiki/NaCl_(software)}
%    - we could use this as our encryption library: this is the same as that used
%      by Tox
%   
%    Implementations of DHTs
%        Kademlia:
%        \url{https://github.com/savoirfairelinux/opendht/}
    
    \section{Additional Details}
    Here, we list some of the technical considerations we will have to resolve
    in our implementation project:
    \begin{itemize}
        \item How many nodes store the data of a single object?  We can start
            with a single node on a network, but what do we want to achieve?
            What degree of redundancy can be achieved and what overhead does
            this add?
        \item Is an entire object stored on each node, or do we distribute a
            single object in incomplete parts through the network?
        \item How large is the address space: how large are the hash keys?  128
            bits?  160 bits?  We may not need to have a large address space for
            our implementation, but just what are the consequences of this
            choice?
        \item What kind of hash function do we use?  Do we use a
            cryptographically secure hash function such as SHA-1 or SHA-2 or do
            we use an unsecure hash function?  What consequences and benefits do
            different approaches confer?
        \item Do we hash the object's identifier or the object data?  If we hash
            the identifier, how do we enforce unique identifiers?
        \item Any hash function will have collisions.  How do we manage
            collisions?  Do we chain hash contents at the site of storage or do
            we dissallow colliding files?
        \item How does the \textit{CAP} theorem (or \textit{Brewer} Theorem)
            inform our implementation? What does it mean for it to be impossible
            for a distributed system to provide all of Consistency,
            Availability, and Partition Tolerance?
        \item What about the keyspace partitioning?  Is it possible for us to
            avoid the issue of key-space re-mapping (changing the node location
            of data) when adding or removing nodes from the network?\\
            To better grasp this issue, consider the case where we use a direct
            hashing method.  If there are $n$ nodes in the system, then we would
            normally store object $o$ at the node $hash(o) \pmod n$.  When
            someone wanted to look up object $o$, they would find it at node
            $hash(o) \pmod n$.  However, if a node disappears and the number of
            active nodes decreases by one, we can no longer find the object $o$
            at $hash(o) \pmod n$.  The other side of this issue is to address how
            objects stores at a node which dissappears are remapped to active
            nodes.  Two separate possibilities to resolve this issue are
            consistent hashing and rendezvous hashing.\\
            Clearly, the hashing method and design we choose is a central
            component and will have large ramifications for the rest of the
            project.  As best we can, we must keep the hashing technique
            decoupled from the rest of the system in our implementation so as to
            avoid creating unnecessary work.

        \item How does a DHT handle node discovery and linkage? Once a file name
            has been hashed to a node, how do we find that node in the network?
            There are a myriad of algorithms designed to do just that:
            Kademlia\cite{kademlia}, Chord\cite{chord}, CAN,
            Tapestry\cite{tapestry}, and
            Pastry\cite{pastry}, to name a few.\\
            Kademlia uses an XOR function on a GUID (per-node unique identifier)
            to calculate the distance between two nodes, because the XOR
            satisfies the requirements of an ideal distance function:
            \begin{itemize}
                \item The distance from A to itself is 0,
                \item The distance from A to B is the same as the distance from
                    B to A, and
                \item It satisfies the triangle inequality: the sum of the
                    distance from A to B to C is greater than the distance
                    between any two of A, B, or C, unless they lie on a line.
            \end{itemize}
            Applications using Kademlia or slightly modified versions of the
            Kademlia algorithm include torrent clients such as BitTorrent, other
            P2P distributed file systems such as IPFS and Gnutella as well as
            P2P chat, voice, and file-sharing programs like Tox.
    \end{itemize}

    Please see the bibliography for additional reading.
	
    \pagebreak

\bibliographystyle{plain}
\bibliography{bibliography}

	\begin{thebibliography}{10}
		\bibitem{ipfs-site}
            \textbf{\url{https://ipfs.io/}}\\
            The official site of the IPFS project.
		\bibitem{tox-site}
            \textbf{\url{https://tox.chat/}}\\
            The official site of the Tox project.
		\bibitem{bittorrent-wiki}
            \textbf{\url{https://en.wikipedia.org/wiki/BitTorrent}}\\
            The BitTorrent wiki article (sorry.)
		\bibitem{ethereum-rlp}
            \textbf{\url{https://github.com/ethereum/devp2p/blob/master/rlpx.md}}\\
            A reference file on the ethereum github.
		\bibitem{distributed-social-networks}
            \textbf{\url{https://arxiv.org/pdf/1508.05591.pdf}}\\
            An article on the use of DHTs for distributed social network
            implementations.
		\bibitem{kademlia}
		    \textbf{\url{https://pdos.csail.mit.edu/~petar/papers/maymounkov-kademlia-lncs.pdf}}\\
            The original \textbf{Kademlia} algorithm paper.
		\bibitem{chord}
	    	\textbf{\url{https://pdos.csail.mit.edu/papers/chord:sigcomm01/chord_sigcomm.pdf}}\\
            A paper on the DHT protocol \textbf{Chord}.
		\bibitem{tapestry}
	    	\textbf{\url{http://www.srhea.net/papers/tapestry_jsac.pdf}}\\
            A paper on the p2p overlay message routing algorithm \textbf{Tapestry}.
		\bibitem{pastry}
	    	\textbf{\url{https://www.cs.rice.edu/~druschel/publications/Pastry.pdf}}\\
            A protocol for internet overlay (overweb) object location and
            routing in potentially very large WANs.
        \bibitem{medium}
	    	\textbf{\url{https://medium.com/karachain/peer-to-peer-protocols-explained-3b1d947c4600}}\\
            An pop-science article on peer-to-peer protocols.
	\end{thebibliography}
\end{document}
