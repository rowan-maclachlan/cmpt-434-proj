\documentclass[12pt]{report}
\usepackage[T1]{fontenc}  % Font encoding
\usepackage{mathptmx}     % Choose Times font 
\usepackage{microtype}    % Improves line breaks      
\usepackage{blindtext}    % Filler text 
\usepackage{hyperref}     % for urls with /url{}

% For our bibliography
\usepackage[backend=biber, sorting=none, hyperref=true]{biblatex}
\addbibresource{\jobname.bib}

%Margin --- 1 inch on all sides
\usepackage[letterpaper]{geometry} \usepackage{times}
\usepackage{xcolor}
\usepackage{listings}
\usepackage{graphicx}
\usepackage{xparse}
\usepackage[font=small,labelfont=bf]{caption}
\NewDocumentCommand{\codeword}{v}{%
	\texttt{\textcolor{blue}{#1}}%
}
\geometry{top=1.0in, bottom=1.0in, left=1.0in, right=1.0in}
% Listings to show code snippets in the report
\usepackage{listings}
% Use as `\begin{lstlisting}...\end{lstlisting}`
\lstdefinestyle{custom}{ 
    language=Python,
    basicstyle=\small,
    breaklines=true,
    frame=single,
    numbers=none,
    tabsize=2, 
    showspaces=false,
    showstringspaces=false,
    keywordstyle=\bfseries\color{green!40!black},
    commentstyle=\itshape\color{purple!40!black},
    identifierstyle=\color{black},
    stringstyle=\color{orange},
}
\lstset{aboveskip=10pt,belowskip=10pt}
\lstset{style=custom}
% For inline code snippets.  use `\code{var = "Hello!"}`
\definecolor{codegray}{gray}{0.9}
\newcommand{\code}[1]{\colorbox{codegray}{\texttt{#1}}}

\usepackage{indentfirst} % Indent the first paragraph after section headings
\setlength{\parskip}{1em} % Set inter-paragraph spacing
\setlength{\parindent}{2em} % Set paragraph indentation
\usepackage{titlesec}
\usepackage{etoolbox}
%%%%%%%%%%%%%%%%%%%%%%%%%%%%%%%%%%%%%%%%%%%%%%%%%%%%%%%%%%%%%%%%%%%
% Title page
%%%%%%%%%%%%%%%%%%%%%%%%%%%%%%%%%%%%%%%%%%%%%%%%%%%%%%%%%%%%%%%%%%%

\begin{document}
	
	\titlepage{
		\vspace*{150px} % Push down title
		\centering % Centre title
		{\large \sc A description of project concepts, milestones, and goals\\}
		\vspace*{30px} % Push down sub text
		\hrulefill \\
		\vspace*{20px}
		{\Huge \uppercase{Kademlia\\}} % Title
		\vspace*{20px}
		\hrulefill \\
		\vspace*{30px}
		February 27, 2019\\
		\vspace*{180px}
		
		\raggedright
		\hfill
		\textit{CMPT 434: Computer Networks} \\
		\vspace*{8px}
	    Christopher	Mykota-Reid
		\hfill
		Derek Eager\\
		Rowan MacLachlan\\
	}
	
	\pagebreak % Put title on its own page
	
	\tableofcontents
	
	\pagebreak
	
    \section{Proposal Requirements:}
    For an implementation project, you should:
    \begin{itemize}
        \item Describe what concept or issue you intend to study. 
        \item Describe the objectives of your investigation. 
        \item Give a brief plan of what you will do to investigate the issue. 
        \item Describe expected measurements or criteria by which you intend to
            evaluate whether you've succeeded at your objective.
    \end{itemize}

    \section{Introduction}
    DHTs are similar in principle to a regular hash table in that they
    provide the structure for fast retrieval of an object by referencing the
    location of the object's hash value.  However, a distributed hash table is
    stored across nodes in a network, and this introduces a variety of
    complications involving node-lookup (finding the location of the stored data
    from its hash in the network), ensuring data redundancy (what happens if a
    node goes offline?), and data retrieval.  Although this technology is not
    \textit{new} by any standard, its continued use and development in various
    fields of application makes it a suitable candidate for an implementation
    project.
    
    \section{Objectives}
    We hope that by implementing a very simple version of a DHT and by demonstrating
    its use on a small network we will become familiar with the issues
    overcome by the more robust and mature DHT implementations.  As referenced above,
    these issues are myriad and often difficult to solve, but not impossible.
    By implementing an incremental plan to gradually roll out more complex features, new
    features, or re-implement sub-systems of our application, we will minimize
    risk so as to end up with a viable end-product for our presentation, while
    also taking full advantage of every learning opportunity which presents
    itself.

    \section{Directives}
    The rough plan for our implementation project will be as follows:
    \begin{enumerate}
        \item Collect and collate relevant technical (algorithm descriptions,
            source-code) and theoretical sources (papers) for the implementation
            of DHTs.  This has already begun.
        \item Divide the DHT into aspects.  Identify concerns such as hashing,
            security, node discovery, redundancy, etc.  This has already begun.
        \item Once the separate issues and their complexities are understood,
            design a minimum-viable product for our implementation.  This MVP
            should be modular and take advantage of as many pre-written
            libraries as are available, while ignoring any details of a DHT
            beyond its barest and least robust implementation.  It should also
            take into consideration the test application we have in
            mind.\footnote{Although it may not result in the most general
            implementation, it is more important that the core concepts are
            applied \textit{successfully} in our particular application.}
        \item Once the MVP has been designed, identify aspects of it which can
            be improved.  This will include things such as data redundancy
            (storing hashed data at multiple nodes), caching, and node
            deletion and addition (how is content from deleted nodes rehashed,
            and how is load redistributed on node addition?).
        \item Implement the MVP, keeping in mind the fact that a
            modular design will make feature improvement and addition easier
            than it would be in a poorly constructed monolithic design.
        \item Once the MVP is implemented and we move onto more advanced
            features, we have to consider the metrics and factors by which we
            measure the effectiveness of our implementation.  What are its
            drawbacks and restrictions?  How efficient is it at finding data on
            the overweb (participating nodes on the internet)?  How efficiently is
            data stored and updated?
    \end{enumerate}

    \section{Implementation}
        \subsection{Use}
            Using the program is relatively simple.  The project folder, once
            downloaded (either provided by the students or downloaded from the
            GIT repository, which is currently private) has a simple
            structure.  The source code is in the \code{kademlia} folder.  So
            long as the user has a python3.7 (or greater) environment,
            everything should work.
            \begin{itemize}
                \item Run \code{make init}. This will install the application
                    dependencies.
                \item run \code{python3 kad.py}.  This will launch a single
                    Kademlia node.  To see application use, run the program
                    without arguments.
                \item Use command \code{set}, \code{get}, \code{ping}, and
                    \code{inspect} to interact with the DHT.
            \end{itemize}
            Please view the README for further details.
        \subsection{High-Level Overview and Tooling}
            \subsubsection{RPC library}
                The Kademlia paper implies that the protocol is implemented
                with datagrams\cite{kademlia}.  Typically, clients operating
                behind a NAT (Network Address Translator) do not need to worry
                about querying servers outside the NAT, as the Network Address
                Translator maintains a table correlating incoming messages to
                hosts on its network.  However, on peer-to-peer systems,
                clients must also act as servers, and accept requests from
                other nodes on the overlay network despite being located behind
                a NAT.  This requires the use of a NAT traversal technique such
                as Hole-Punching or SOCKS (Socket Secure) to overcome the
                difficulties inherent in routing requests through NATs.  This
                makes application use on LANs behind a router address difficult
                without port-forwarding and configuration overhead.  Because of
                this, our application was tested only on a LAN.

                Our primary investigation showed other Kademlia
                implementations using the core python library asyncio to manage
                callback procedures for asyncronous code execution. While the
                semantics of asynchronous RPC might make implementation
                \textit{more} difficult than for a simple synchronous I/O
                implementation (the python socket module directly, for
                example), implementing asynchronous UDP RCP with asyncio is
                very simple.  In fact, we used a 3rd party library called
                rpcudp which overlays functionality for remote procedure calls
                onto a datagram communication protocol - all
                asynchronously!\cite{rpcudp}

                As described on the asyncio Read-The-Docs page: "asyncio is
                often a perfect fit for IO-bound and high-level structured
                network code." This is exactly the Kademlia
                use-case.\cite{asyncio}

            \subsubsection{hashing}
                Kademlia specifies the use of the 160-bit SHA-1 hash for data,
                and 160-bit node IDs and 160
                \code{k-buckets}\ref{bucket_design}, to accompany that.
                This parameter of ID and hash length, however, is not mandated.
                The python library hashlib provides an array of hashing
                functions, cryptographic and otherwise, including the SHA-1
                hash.  This library can be used behind a small wrapper, and
                hash/ID size can be trimmed to the length specified in the
                simulation parameters.  This value could initially be set quite
                small to increase ease-of-testing and provide more
                comprehensible and tractable output during development stages.
                It can easily be changed later.
\begin{lstlisting}[label=hashing]
def hash_function(data):
    """
    Hash the data to a byte array of length p.params[B] / 8

    TODO: How do we truncate this large integer?  mod?  or mask?
        what would we mod by?

    data : binary data
    
    int 
        A hash of length p.params[B] / 8 in hexadecimal string form.
    """
        
    return int(hashlib.sha1(data).hexdigest(), 16) & get_mask() 
\end{lstlisting}
            
            As shown here, the returned digest is AND'ed with a mask --- this
            mask is the length of the system-wide hash/ID bit length value ---
            called `B' in the Kademlia paper.  This mask then truncates the
            sha1 hash of the data to a value between 0 and $2^b-1$.  This
            allows the value of `B' to be effortlessly changed for more
            practical use of the network or for larger-scale simulations.

            \subsubsection{Routing Table Design} \label{routing_table_design}
                Study of different Kademlia
                implementations\cite{implementation_01}\cite{implementation_02}\cite{implementation_03}\cite{implementation_04}
                reveal that most implementations implement their routing table
                in a way very similar to that described in the Kademlia white
                paper\cite{kademlia}.  This involves creating nodes with only a
                single \code{k-bucket}\ref{bucket_design}.  As the k-bucket
                reaches capacity, it is then split into two and its contents
                distributed among the two resultant buckets according to their
                proximity to the node that owns the routing table.  In this
                sense, a routing table may not actually have as many
                \code{k-bucket}s
                in memory as there are bits in the keyspace.  Naturally, this
                is far more space efficient than the implementation we opted
                for.

                As the maximum number of \code{k-bucket}s on any particular
                node is simply the length of the ID space, they can all be
                created at once, as such:
\begin{lstlisting}[label=RoutingTable.__init__]
self.buckets = [ KBucket(k) for _ in range(b) ]
\end{lstlisting}
                Instead of attempting to insert nodes into a \code{k-bucket}
                which may or may not be full and which may result in further
                operations on the data structure, nodes can be inserted
                directly into the \code{k-bucket} corresponding to their
                distance from the routing table's owner, as such:
\begin{lstlisting}[label=RoutingTable.add]
return self.get_bucket(contact.getId()).add(contact)
\end{lstlisting}
                where the method \code{RoutingTable.get\_bucket(id)} calculates
                the correct bucket index by referencing the most significant
                bit of the distance between the ID of the routing table's owner
                and the ID of the contact we are trying to add to the routing
                table, like so:
\begin{lstlisting}[label=RoutingTable.get_bucket]
distance = self.id ^ id
index = 0
while distance > 1:
    # Count the index of the largest bit in the distance
    # The distance will be zero after 'bit' many shifts.
    distance = distance >> 1
    index += 1
return self.buckets[index]
\end{lstlisting}
                Naturally, \code{k-bucket}s can be retrieved in a similar
                manner.
                
                Although this approach is less space efficent than creating
                \code{k-bucket}s only as needed, the overhead is fixed and
                relatively small.  Ultimately, this decision was made because
                it was believed to be simpler.

            \subsubsection{Bucket Design} \label{bucket_design}
                The \code{k-bucket} is a container
                that holds contact information for other nodes on the Kademlia
                network.  Each \code{k-bucket} is created with a maximum
                capacity of \code{k}, a system-wide parameter that controls for
                such things as \code{k-bucket} size.  In our implementation, it
                is simply a light wrapper for a list into which contacts are
                stored.
                
                As described by the white
                paper, \code{k-bucket}s, once full, need to make evaluations
                about whether new contacts are inserted in the \code{k-bucket}
                or not after it has reached capacity.  When a new node is
                discovered but its \code{k-bucket} is full, it is only added if
                the node at the head of the list is unresponsive.  Otherwise,
                the new node is simply discarded, and the responsive node is
                moved from the head to the tail of the list.

                There are a variety of reasons for this behaviour.  First, the
                Kademlia authors illustrate that nodes which have existed on
                the network for a long time are statistically more likely to
                continue being active on the network than new nodes.
                Therefore, the integrity of the network as a whole is improved
                by prioritizing these old nodes.  Secondly, this behaviour
                makes the network automatically resistant to Denial-of-Service
                attacks: the network cannot be brought down by flooding it with
                new nodes, because the old network will not replace legitimate
                nodes that already exist in its routing table unless they
                become unresponsive.

                Unfortunately, our implementation does not track node
                responsiveness, as the minimum-viable-product was simpler if
                new nodes are simply added (because they are known to be
                active) and old nodes are kicked out of the routing table.

\begin{lstlisting}[label=KBucket.add]
if contact in self.contacts:
    # If this contact already appears in the list, move it to the back
    # of the list
    self.contacts.remove(contact)
    self.contacts.append(contact)
elif not self.full():
    # If the contact doesn't appear in the list, append it.
    self.contacts.append(contact)
else:
    # If the list is full, remove the oldest contact before appending
    # the new one.
    self.contacts.popleft()
    self.contacts.append(contact)
return True
\end{lstlisting}
            Because most operation on the \code{k-bucket} involve adding and
            removing nodes from the beginning and end of the list, a
            special-use data structure was employed to optimize the efficiency
            of these operations.  Python provides a structure called a `deque'
            (pronounced `deck') that implements constant-time additions and
            removals from the head and tail of the list.

        \subsection{Kademlia Remote Procedure Calls}
            \subsubsection{Store}
            \subsubsection{Find Value}
            \subsubsection{Find Node}
            \subsubsection{Ping}
            \subsubsection{Finding Nodes on the Network}

    \section{Measurements and Criteria}
    As referenced in the implementation plan, there are quantitative
    measurements available when considering the success or failure of our
    implementation.  Comparing these values to those achieved by more mature
    DHT implementations will provide valuable insight into the challenges posed
    by DHTs, and should be included in our final report.  However, the most
    valuable metric will be the realization of a working MVP.  This may be
    something as simple as the observed and correct distributed storage of one
    or more files on the personal computers of the researchers (Chris and
    Rowan) or a more glorious success - or, perhaps, the real success is the
    friends we made along the way.

    \section{Additional Details}
    Here, we list some of the technical considerations we will have to resolve
    in our implementation project:
    \begin{itemize}
        \item How many nodes store the data of a single object?  We can start
            with a single node on a network, but what do we want to achieve?
            What degree of redundancy can be achieved and what overhead does
            this add?
        \item Is an entire object stored on each node, or do we distribute a
            single object in incomplete parts through the network?
        \item How large is the address space: how large are the hash keys?  128
            bits?  160 bits?  We may not need to have a large address space for
            our implementation, but just what are the consequences of this
            choice?
        \item What kind of hash function do we use?  Do we use a
            cryptographically secure hash function such as SHA-1 or SHA-2 or do
            we use an unsecure hash function?  What consequences and benefits do
            different approaches confer?
        \item Do we hash the object's identifier or the object data?  If we hash
            the identifier, how do we enforce unique identifiers?
        \item Any hash function will have collisions.  How do we manage
            collisions?  Do we chain hash contents at the site of storage or do
            we dissallow colliding files?
        \item How does the \textit{CAP} theorem (or \textit{Brewer} Theorem)
            inform our implementation? What does it mean for it to be impossible
            for a distributed system to provide all of Consistency,
            Availability, and Partition Tolerance?
        \item What about the keyspace partitioning?  Is it possible for us to
            avoid the issue of key-space re-mapping (changing the node location
            of data) when adding or removing nodes from the network?\\
            To better grasp this issue, consider the case where we use a direct
            hashing method.  If there are $n$ nodes in the system, then we would
            normally store object $o$ at the node $hash(o) \pmod n$.  When
            someone wanted to look up object $o$, they would find it at node
            $hash(o) \pmod n$.  However, if a node disappears and the number of
            active nodes decreases by one, we can no longer find the object $o$
            at $hash(o) \pmod n$.  The other side of this issue is to address how
            objects stores at a node which dissappears are remapped to active
            nodes.  Two separate possibilities to resolve this issue are
            consistent hashing and rendezvous hashing.\\
            Clearly, the hashing method and design we choose is a central
            component and will have large ramifications for the rest of the
            project.  As best we can, we must keep the hashing technique
            decoupled from the rest of the system in our implementation so as to
            avoid creating unnecessary work.

        \item How does a DHT handle node discovery and linkage? Once a file name
            has been hashed to a node, how do we find that node in the network?
            There are a myriad of algorithms designed to do just that:
            Kademlia\cite{kademlia}, Chord\cite{chord}, CAN,
            Tapestry\cite{tapestry}, and
            Pastry\cite{pastry}, to name a few.\\
            Kademlia uses an XOR function on a GUID (per-node unique identifier)
            to calculate the distance between two nodes, because the XOR
            satisfies the requirements of an ideal distance function:
            \begin{itemize}
                \item The distance from A to itself is 0,
                \item The distance from A to B is the same as the distance from
                    B to A, and
                \item It satisfies the triangle inequality: the sum of the
                    distance from A to B to C is greater than the distance
                    between any two of A, B, or C, unless they lie on a line.
            \end{itemize}
            Applications using Kademlia or slightly modified versions of the
            Kademlia algorithm include torrent clients such as BitTorrent, other
            P2P distributed file systems such as IPFS and Gnutella as well as
            P2P chat, voice, and file-sharing programs like Tox.
    \end{itemize}

    Please see the bibliography for additional reading.
	
    \pagebreak

\nocite{*}
\printbibliography

\end{document}
